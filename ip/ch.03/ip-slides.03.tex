
%%%%%%%%%%%%%%%%%%
%
%
%%%%%%%%%%%%%%%%%%
\begin{slide}{Agenda}
\tableofcontents
\end{slide}


\section{Simple Processing}

%%%%%%%%%%%%%%%%%%%%%%%%%%%%%%%%%%%
%
%
%%%%%%%%%%%%%%%%%%%%%%%%%%%%%%%%%%%
\begin{slide}{Review: Image as matrix}

\begin{itemize} %{}{}
\small
\item<2-> An image matrix ($N\times M$):
{\scriptsize
\begin{eqnarray*} 
{\bf A}=
\left.
\left[
    \begin{array}{lllll}
    A(0,0)&A(0,1)&A(0,2)&\ldots &A(0,M-1) \\
    A(1,0)&A(1,1)&A(1,2)&\ldots &A(1,M-1) \\ \vdots \\
    A(N-1,0)&A(N-1,1)&A(N-1,2)&\ldots &A(N-1,M-1)
    \end{array}
\right] %\
%\ 
\right\} N\ \mbox{rows} 
\end{eqnarray*}
}
      
\item<3->  $A(i,j)\in \{0,1,\ldots,255\}$.
\item<4->  $A(i,j)$:
    \begin{itemize}
    \small
        \item \blue{Matrix case:}
        The matrix element $(i,j)$ with value $A(i,j)$.
        \item \blue{Image case:}
        The pixel $(i,j)$ with value $A(i,j)$.
        \item Will use both terminologies.
    \end{itemize}

\end{itemize}
\onslide<5->
\begin{alertblock}{Remember}
In Matlab the indices $(i,j)$ start from 1 as opposed to 0
\end{alertblock}

\end{slide}

\subsection{Transpose}
%%%%%%%%%%%%%%%%%%
%
%
%%%%%%%%%%%%%%%%%%
\begin{slide}{Simple Processing - Transpose}
\scriptsize
The transpose image ${\bf B}$ $(M\times N)$ of ${\bf A}$ $(N\times M)$
can be obtained
as $B(j,i)=A(i,j)$\\ $(i=0,\ldots,N-1,\ j=0,\ldots,M-1)$.
\slidefig[0.7]{transp.jpg}

\begin{code}[8]{}
\begin{tabbing}
for \= i = 1:512 \\
\>	for \= j = 1:512 \\
\> \>	B(j,i) = A(i,j) \\
\> end \\
end
\end{tabbing}
\end{code}
\end{slide}

\subsection{Flip}

%%%%%%%%%%%%%%%%%%
%
%
%%%%%%%%%%%%%%%%%%
\begin{slide}{Simple Processing - Vertical Flip}
\scriptsize
The vertical flipped image ${\bf B}$ $(N\times M)$ of ${\bf A}$ $(N\times M)$
can be obtained
as $$B(i,M-1-j)=A(i,j)$$ $(i=0,\ldots,N-1,\ j=0,\ldots,M-1)$.

\slidefig[0.7]{flip.jpg}

\begin{code}[8]{}
\begin{tabbing}
for \= i = 1:512 \\
\>	for \= j = 1:512 \\
\> \>   B(i,512+1-j)=A(i,j) \\
\> end \\
end \\
%$>>\ $ image($B$); \\
%$>>\ $ colormap(gray(256)); \\
%$>>\ $ axis('image');
\end{tabbing}
\end{code}

\end{slide}

\subsection{Cropping}
%%%%%%%%%%%%%%%%%%
%
%
%%%%%%%%%%%%%%%%%%
\begin{slide}{Simple Processing - Cropping}
\scriptsize
The cropped image ${\bf B}$ $(N_1\times N_2)$ of ${\bf A}$ $(N\times
M)$,
starting from $(n_1,n_2)$, can be obtained
as $B(k,l)=A(n_1+k,n_2+l)$ $(k=0,\ldots,N_1-1,\ l=0,\ldots,N_2-1)$.

\slidefig[0.7]{crop.jpg}

\begin{code}[8]{}
\begin{tabbing}
for \= k = 0 : 64 - 1 \\
\>  for \= l = 0 : 128 - 1 \\
\>	\>	B(k+1, l+1) 1= A(255+k+1, 255+l+1); \% n1=n2=255 N1=64,N2=128 \\
\>	end \\
end
%$>>\ $ image($B$); \\
%$>>\ $ colormap(gray(256)); \\
%$>>\ $ axis('image');
\end{tabbing}
\end{code}

\end{slide}


%%%%%%%%%%%%%%%%%%
%
%
%%%%%%%%%%%%%%%%%%
\begin{slide}{Cropping as Matlab function}
\begin{code}[8]{}
function [B] = mycrop(A, n1, n2, N1, N2) \\
\% mycrop.m \\
\% [B] = mycrop(A, n1, n2, N1, N2) \\
\% Crops image A from location n1,n2 \\
\% with size N1,N2 
\begin{columns}
\begin{column}{0.5\textwidth}
\begin{tabbing}
\% using loops\\
for \= k = 0 : N1-1 \\
\>    for \= l = 0 : N2 - 1 \\
\>	 \>    B(k+1,l+1) = A(n1+k+1,n2+l+1); \\
\>    end \\
end \\  
\end{tabbing}
\end{column}
\begin{column}{0.5\textwidth}
\% without loops\\
B(1:N1,1:N2)=A(n1+1:n1+N1,n2+1:n2+N2); 
\end{column}
\end{columns}

\end{code}
\end{slide}

\section{Simple Statistics}

\subsection{Sample Mean \& Variance}
%%%%%%%%%%%%%%%%%%
%
%
%%%%%%%%%%%%%%%%%%
\begin{slide}{Simple Statistics: Sample Mean \& Variance}
\begin{titlelist}{}{}
\small
\item<2-> 
\darkred{Sample mean} ($m_A$) of an image ${\bf A}$
($N\times M$):
\vspace*{-6pt}
\begin{eqnarray}
m_A={\displaystyle \sum_{i=0}^{N-1}\sum_{j=0}^{M-1} A(i,j) \over N M}
\end{eqnarray}

\item<3-> 
\darkred{Sample variance} ($\sigma^2_A$) of ${\bf A}$:
\vspace*{-6pt}
\begin{eqnarray}
\sigma^2_A={\displaystyle \sum_{i=0}^{N-1}\sum_{j=0}^{M-1} (A(i,j)-m_A)^2 \over N M}
\end{eqnarray}

\item<4-> 
\darkred{Sample standard deviation},
\vspace*{-6pt}
$$\sigma_A=\sqrt{\sigma^2_A}$$.

\end{titlelist}
\end{slide}



\subsection{Histogram}
%%%%%%%%%%%%%%%%%%
%
%
%%%%%%%%%%%%%%%%%%
\begin{slide}{Simple Statistics: Histogram}
\begin{titlelist}{}{}
\small
\item<2-> 
Let $S$ be a set and define $\#S$ to be the cardinality of this
set,\\
i.e., $\#S$ is the number of elements of $S$.

\item<3-> 
\darkred{Histogram} $h_A(l)$ $(l=0,\ldots,255)$ of the image
${\bf A}$ is defined as:
{\scriptsize 
\begin{eqnarray}
h_A(l)&=&\#\{(i,j)\ |\ A(i,j)=l,\ i=0,\ldots,N-1,\ j=0,\ldots,M-1\}
\end{eqnarray}
}
\item<4-> 
Note that:
\begin{eqnarray}
\sum_{l=0}^{255} h_A(l)=\mbox{\scriptsize Number of pixels in}\ {\bf A}
\end{eqnarray}

\end{titlelist}

\end{slide}


%%%%%%%%%%%%%%%%%%
%
%
%%%%%%%%%%%%%%%%%%
\begin{slide}{Histogram calculation}

\begin{columns}
\begin{column}{0.5\textwidth}

\begin{code}[8]{}
\begin{tabbing}
>>  \=  h = zeros(256,1);    \\
>>  \> for  \= l = 0:255            \\
    \>      \> for  \= i = 1:N               \\
    \>      \>       \> for \=  j = 1:M                \\
    \>	\>	\>	\>      if  \= (A(i,j)==l)        \\
    \>	\>	\>	\>  \>      h(l+1)=h(l+1)+1;   \\
    \>	\>	\>	\>      end  \\
    \>      \>  \> end  \\
    \>      \> end  \\
    \> end  \\
>>  \> bar(0:255,h);   \\
\end{tabbing}
\end{code}

\end{column}
\begin{column}{0.5\textwidth}

\begin{code}[8]{}

\begin{tabbing}
>>  \=  h = zeros(256,1); \\
>>	\>	for  \= l = 0:255   \\
	\>  \>   h(l+1) = sum(sum(A==l)); \\
	\>  end \\
>> \> bar(0:255,h);  \\
\end{tabbing}  
\end{code}
\end{column}
\end{columns}
\end{slide}


%%%%%%%%%%%%%%%%%%
%
%
%%%%%%%%%%%%%%%%%%
\begin{slide}{Example -- \#1}

\slidefig[0.8]{hist1.jpg}\\
\centering
${\bf A}$ \hspace{0.4\textwidth} $h_A(l)$

\end{slide}

%%%%%%%%%%%%%%%%%%
%
%
%%%%%%%%%%%%%%%%%%
\begin{slide}{Example -- \#2}

\slidefig[0.8]{hist2.jpg}\\
\centering
${\bf A}$ \hspace{0.4\textwidth} $h_A(l)$

\end{slide}

%%%%%%%%%%%%%%%%%%
%
%
%%%%%%%%%%%%%%%%%%
\begin{slide}{Example -- \#3}
\hypertarget{origlenna}{
\slidefig[0.8]{histlen.jpg}
}\\
\centering
${\bf A}$ \hspace{0.4\textwidth} $h_A(l)$

\end{slide}

\section{Point processing}
%%%%%%%%%%%%%%%%%%
%
%
%%%%%%%%%%%%%%%%%%
\begin{slide}{Point Processing}
\begin{titlelist}{}{}
\small
\item<2-> 
We will now utilize a ``function'' $g(l)\ (l=0,\ldots,255)$
to generate a new image ${\bf B}$ from a given image ${\bf A}$
via:
{\scriptsize
\begin{eqnarray}
B(i,j)=g\left(A(i,j)\right),\ \ \ i=0,\ldots,N-1,\ j=0,\ldots,M-1
\end{eqnarray}
}
\item<3-> 
The function $g(l)$ operates on each image pixel
or each image \darkred{point} independently.

\item<4-> 
In general the resulting image $g(A(i,j))$ may not
be an image matrix, i.e., it may be the case that
$g(A(i,j)) \not\in \{0,\ldots,255\}$ for some $(i,j)$.

\item<5-> 
Thus we will have to make sure we obtain an image ${\bf B}$
that is an image matrix.

\item<6-> 
The histograms $h_A(l)$ and $h_B(l)$ will play important
roles.

\end{titlelist}
\end{slide}

\subsection{Identity point function}
%%%%%%%%%%%%%%%%%%
%
%
%%%%%%%%%%%%%%%%%%
\begin{slide}{Identity Point Function}
\begin{titlelist}{}{}

\item<2-> 
Let $g(l)=l\ (l=0,\ldots,255)$.
\begin{eqnarray*}
B(i,j)&=&g(A(i,j)),\ \ \ i=0,\ldots,N-1,\ j=0,\ldots,M-1 \\
    &=&A(i,j)
\end{eqnarray*}

\item<3-> In this case $g(A(i,j)) \in \{0,\ldots,255\}$ so no further
processing is necessary to ensure ${\bf B}$ is an image matrix.

\item<4-> Note also that ${\bf B=A}$ and hence $h_B(l)=h_A(l)$.

\end{titlelist}

\end{slide}

\subsection{Digital Negative}
%%%%%%%%%%%%%%%%%%
%
%
%%%%%%%%%%%%%%%%%%
\begin{slide}{Digital Negative}

\onslide<2->
Let $g(l)=255-l\ (l=0,\ldots,255)$.
\onslide<3->
\begin{eqnarray*}
B(i,j) &=& g(\text{\hyperlink{origlenna}{$A(i,j)$}}),\ \ \ i=0,\ldots,N-1,\ j=0,\ldots,M-1 \\
    &=& 255- \text{\hyperlink{origlenna}{$A(i,j)$}}
\end{eqnarray*}

\slidefiganim[1-3]{height=0.5\textheight}{spacer.jpg}
\centering\slidefiganim[4-]{height=0.5\textheight}{histneglen.jpg}

\end{slide}


%%%%%%%%%%%%%%%%%%
%
%
%%%%%%%%%%%%%%%%%%
\begin{slide}{Digital Negative (cont'd)}
\slidefig[0.8]{h_len_nlen.jpg}
\centering
In this case it is easy to see that $h_B(255-l)=h_A(l)$ or $h_B(g(l))=h_A(l)$

\end{slide}

%%%%%%%%%%%%%%%%%%
%
%
%%%%%%%%%%%%%%%%%%
\begin{slide}{Histogram of Point Processed Images}
\begin{titlelist}{}{}

\item<2-> 
For a given point function $g(l)$, ``the inverse point
function'' $g^{-1}(k)$ may not exist.

\item<3-> \red{So in general $h_B(g(l))\not = h_A(l)$}.

\item<4-> 
Let $\blue{S_{g^{-1}(k)}}=\{l\ |\ g(l)=k,\ l=0,\ldots,255\}$
$(k=0,\ldots,\ 255)$.

\item<5-> Then:
\begin{eqnarray}
h_B(k)=\sum_{l\in \blue{S_{g^{-1}(k)}}} h_A(l),\ \ \ \ k=0,\ldots,255
\end{eqnarray}

\item<6-> 
We must learn how to calculate and sketch $h_B(l)$
given $h_A(l)$ and the point function $g(l)$.

\end{titlelist}

\end{slide}


%%%%%%%%%%%%%%%%%%
%
%
%%%%%%%%%%%%%%%%%%
\begin{slide}{Square-root point function}
\begin{titlelist}{}{}
\small

\item<2-> 
Let $g(l)=\sqrt{l}\ (l=0,\ldots,255)$.
\begin{eqnarray*}
B(i,j)&=&g(A(i,j)),\ \ \ i=0,\ldots,N-1,\ j=0,\ldots,M-1 \\
    &=&\sqrt{A(i,j)}
\end{eqnarray*}

\item<3-> In this case $g(A(i,j)) \not \in \{0,\ldots,255\}$ and {\em we must
ensure} ${\bf B}$ is an image matrix by further processing.

\item<4-> 
We can try ``rounding'' $g(A(i,j))$ to integers by defining
a new point function $g_2(l)=\mbox{round}(g(l))$:
\begin{eqnarray*}
B(i,j)&=& g_2(A(i,j)),\ \ \ i=0,\ldots,N-1,\ j=0,\ldots,M-1 \\
    &=&\mbox{round}(\sqrt{A(i,j)})\ \ (>>\ B=\mbox{round}(A.\hat{\ }(.5));)
\end{eqnarray*}
but that
will not exactly yield what we want % as we shall see.

\end{titlelist}

\end{slide}

\subsection{Contrast Stretching}
%%%%%%%%%%%%%%%%%%
%
%
%%%%%%%%%%%%%%%%%%
\begin{slide}{\hypertarget{contrast}{Contrast Stretching}}

{\small
\begin{eqnarray}
g(l)=\left\{ \begin{array}{ll}
\alpha_1 l, & 0\leq l < a_1 \\
\alpha_2 (l-a_1)+ \alpha_1 a_1, & a_1 \leq l < a_2  \\
\alpha_3 (l-a_2)+ (\alpha_2 (a_2-a_1)+\alpha_1 a_1), & a_2\leq l
\leq
255
\end{array}\right.
\end{eqnarray}
}
\begin{columns}
\begin{column}{0.6\textwidth}
\insertfig[height=0.6\textheight]{contrast.jpg}
\end{column}
\begin{column}{0.4\textwidth}
\scriptsize
\begin{tabular}{ll}
$\alpha_i > 1 \Rightarrow$ & Range Stretching \\
$\alpha_i < 1 \Rightarrow$ & Range Compression
\end{tabular}
\end{column}
\end{columns}

\end{slide}


%%%%%%%%%%%%%%%%%%
%
%
%%%%%%%%%%%%%%%%%%
\begin{slide}{Example - Square-root}
\vspace*{-16pt}
\slidefig[0.8]{sqrt1.jpg}
\vspace*{-22pt}
\slidefig[0.8]{sqrt2.jpg}
\end{slide}


%%%%%%%%%%%%%%%%%%
%
%
%%%%%%%%%%%%%%%%%%
\begin{slide}{Piecewise Linear, ``Continuous'' Point Functions}
\vspace*{-6pt}
\begin{titlelist}{}{}
\item<2-> 
Let $\alpha_0=a_0=0$.
\vspace{-6pt}
{\scriptsize
\begin{eqnarray}
g(l)=\left\{ \begin{array}{ll}
\alpha_1 l, & 0\leq l < a_1 \\
\alpha_2 (l-a_1)+ \alpha_1 a_1, & a_1 \leq l < a_2  \\
\vdots \\
\alpha_i (l-a_{i-1})+ (\sum_{j=1}^{i-1} \alpha_j (a_j-a_{j-1})), & a_{i-1}\leq l
< a_i \\
\vdots
\end{array}\right.
\end{eqnarray}
}
\item<3-> $|\alpha_i|<1 \Rightarrow$ \red{Range Compression}.
\item<4-> $|\alpha_i|>1 \Rightarrow$ \blue{Range Stretching}.

\item<5-> 
Assuming $0\leq g(l) \leq 255$ $(l=0,\ldots,255)$,
 affect point function by incorporating the $\mbox{round(\ldots)}$
 function when necessary, i.e.,
 $g_2(l)=\mbox{round}(g(l)):$
\vspace*{-3pt} 
{\scriptsize
\begin{eqnarray*}
B(i,j)=g_2(A(i,j));
\end{eqnarray*}
}
\vspace*{-18pt} 
\end{titlelist}

\end{slide}



%%%%%%%%%%%%%%%%%%
%
%
%%%%%%%%%%%%%%%%%%
\begin{slide}{} %{\small ``Discontinuous'' Point Function}
\centering\textbf{``Discontinuous'' Point Function}
\vspace*{-4pt}
\slidefig[0.8]{fpeak.jpg}
\end{slide}

\subsection{Normalization}
%%%%%%%%%%%%%%%%%%
%
%
%%%%%%%%%%%%%%%%%%
\begin{slide}{Normalizing an Image}
\begin{titlelist}{}{}

\item<2-> 
Let $mx_A= \max_{(i,j)} A(i,j)$ and $mn_A= \min_{(i,j)}
A(i,j)$.\\

These can be generated in matlab via:

\begin{tabular}{ll}
$>>$&$mx$=max(max($A$)); \\
$>>$&$mn$=min(min($A$));
\end{tabular}

\item<3-> 
The \blue{{\em special normalizing point function}}
{\em for ${\bf A}$} is defined as:
\begin{eqnarray}
\mbox{\red{$g_s^A(l)$}}=\mbox{round}\left({l-mn_A \over mx_A-mn_A}\times
255 \right)
\end{eqnarray}

\item<4-> 
Convention: \\

\blue{\small
${\bf A}\Rightarrow$ {processing} $\Rightarrow\ {\bf B}\Rightarrow$
``{normalize} ${\bf B}$''$\Rightarrow$
$C(i,j)=g_s^B(B(i,j))$
}

\item<5-> 
Note that this will work even if B had negative values!

\end{titlelist}

\end{slide}


%%%%%%%%%%%%%%%%%%
%
%
%%%%%%%%%%%%%%%%%%
\begin{slide}{More Range Stretching/Compression}
\begin{titlelist}{}{}
\small
\item<2-> 
In some cases we wish to view a matrix (image matrix
or otherwise)
having a wide range of values.

\item<3-> 
Let
\begin{eqnarray*}
B(i,j)=\left\{\begin{array}{ll}
A(0,0)+10^6 & i=j=0\\
A(i,j) & \mbox{otherwise}
\end{array}\right.
\end{eqnarray*}
where ${\bf A}$ is an image
matrix $(A(i,j)\in \{0,\ldots,255\})$.

\item<4-> Normalizing ${\bf B}$ using $C(i,j)=g_s^B(B(i,j))$
will show an image $({\bf C})$ that is completely black except for the
pixel $(0,0)$ which will have the value $255$.

\item<5->  This loses all the information in ${\bf C}$ about ${\bf
A}$.

\item<6-> 
The next two slides consider an example which addresses
the problem by point processing.
The resulting point function is sometimes called
an emphasis/de-emphasis function.

\end{titlelist}

\end{slide}



%%%%%%%%%%%%%%%%%%
%
%
%%%%%%%%%%%%%%%%%%
\begin{slide}{Example}
\vspace*{-9pt}
\scriptsize
\begin{itemize}
\item A good example of the ``wide range'' problem
happens when we want to show the {\em 2-D Discrete Fourier
Transform (DFT)} of an image.
\item $>> F=\mbox{round}(\mbox{fftshift}(\mbox{abs}(
\mbox{fft2}(A))));$
\item Let $B(i,j)=log_{10}(F(i,j)+1)$ and
$C(i,j)=g_s^F(F(i,j))$,
$D(i,j)=g_s^B(B(i,j))$.
\end{itemize}
\slidefig[0.5]{log.jpg}
\end{slide}


%%%%%%%%%%%%%%%%%%
%
%
%%%%%%%%%%%%%%%%%%
\begin{slide}{Example (cont'd)}
\slidefig[0.9]{logfft.jpg}
\centering
\scriptsize
Specific range stretching/compression might yield even better
results.
This is an effective tool to visualize things quickly.
Different problems can be tackled in the same spirit
by utilizing point functions based on $log_{10}(\ldots),\
log_{10}(log_{10}(\ldots)),\ e^{\ldots}$ and so on.
\end{slide}


%%%%%%%%%%%%%%%%%%
%
%
%%%%%%%%%%%%%%%%%%
\begin{slide}{Slicing}
\vspace*{-12pt}
\slidefig[0.75]{thres.jpg}
\end{slide}


%%%%%%%%%%%%%%%%%%
%
%
%%%%%%%%%%%%%%%%%%
\begin{slide}{Thresholding}
Binary thresholding $(T=128)$
\begin{eqnarray*}
B(i,j)=\left\{\begin{array}{ll}
0& A(i,j)\leq T \\
255& A(i,j)>T
\end{array}\right.
\end{eqnarray*}
\slidefig[0.8]{binary.jpg}
\end{slide}


%%%%%%%%%%%%%%%%%%
%
%
%%%%%%%%%%%%%%%%%%
\begin{frame}{Assignment} % [allowframebreaks]
\begin{enumerate}
\item Flip your image horizontally and diagonally.
Print the results.
\item Crop a portion of your image that mostly shows
your head.
Print the result and parameters used.
\item Calculate the mean, variance and histogram of your image.
Show the histogram as a bar plot next to your image
(use the {\em subplot} command).
Indicate the size of your image.
%\item Compute the square-root of your image.
%Round it using the {\bf round} function.
%Show the result and its histogram.
%Now normalize the rounded image.
%Show the result and its histogram.
%Briefly compare it to your original, unprocessed image.
%\item Compute the square-root of your image.
%Normalize the result and print it together with its histogram.
%Briefly compare it to the output of (4.) and to your original image.
%Which pixel value ranges got stretched/compressed?
%\item Compute $B(i,j)=log_{10}(A(i,j)+1)$ where ${\bf A}$ is your
%image.
%${\bf B}$ is a general matrix and not an image matrix.
%Generate an image matrix from ${\bf B}$ by normalizing it.
%Show the normalized image next to the original.
%Briefly describe the changes.
%Anything that can be seen better?
%Which pixel value ranges got stretched/compressed?

\end{enumerate}
\end{frame}
