
%%%%%%%%%%%%%%%%%%
%
%
%%%%%%%%%%%%%%%%%%
\begin{slide}{Agenda}
\tableofcontents
\end{slide}

%%%%%%%%%%%%%%%%%%%%%%%%%%%%%%%%%%%%%%%%%%%%%%%%%%%%%%%%%%%%%%%%%%%%%%%%
%
%
%%%%%%%%%%%%%%%%%%%%%%%%%%%%%%%%%%%%%%%%%%%%%%%%%%%%%%%%%%%%%%%%%%%%%%%%
\section{Introduction}

\subsection{Dynamic range and visibility}
%%%%%%%%%%%%%%%%%%%%%%%%%%%%%%%%%%%
%
%
%%%%%%%%%%%%%%%%%%%%%%%%%%%%%%%%%%%
\begin{slide}{Dynamic range, visibility, and contrast enhancement}
\vspace{-3ex}
\slidefig[0.55]{scale.jpg}
\vspace{-5ex}
\begin{titlelist}{}{}
\small
\item<2-> 
Contrast enhancing point functions we have discussed
earlier expand the dynamic range occupied by certain 
``interesting'' pixel
values in the input image.

\item<3-> 
These pixel values in the input image may be
difficult to distinguish and the goal of contrast enhancement
is to make them ``more visible'' in the output image.

\item<4-> 
Don't forget we have a limited dynamic range $(0-255)$ at
our disposal.

\end{titlelist}
\end{slide}


\subsection{Point functions and Histograms}
%%%%%%%%%%%%%%%%%%%%%%%%%%%%%%%%%%%
%
%
%%%%%%%%%%%%%%%%%%%%%%%%%%%%%%%%%%%
\begin{slide}{Point functions and Histograms}
\begin{titlelist}{}{}

\item<2-> 
In general a \darkred{point operation/function} $B(i,j)=g(A(i,j))$
results in a new histogram $h_B(l)$ for the output image
that is different from $h_A(l)$.

\item<3-> 
The relationship between $h_B(l)$ and $h_A(l)$ may not be
straightforward as we have already discussed in previous lecture.

\item<4-> 
We {\em must} learn how to calculate $h_B(l)$ given $h_A(l)$
and the point function $g(l)$:

\begin{itemize}
\item<5-> 
\red{Exactly:} Usually via writing a matlab script that
computes $h_B(l)$ from $h_A(l)$ and $g(l)$.

\item<6-> 
\red{Approximately:} By sketching  $h_B(l)$
given the sketches for $h_A(l)$ and $g(l)$.
\end{itemize}

\end{titlelist}
\end{slide}


%%%%%%%%%%%%%%%%%%%%%%%%%%%%%%%%%%%
%
%
%%%%%%%%%%%%%%%%%%%%%%%%%%%%%%%%%%%
\begin{slide}{``Unexpected'' Effect of some Point Functions}
\vspace*{-3ex}
\begin{columns}
\begin{column}{0.6\textwidth}
\slidefig[0.7]{round.jpg}
\end{column}
\begin{column}{0.4\textwidth}
\begin{itemize}
\small
\item<1-> ${\bf B}$ has $\sim 10$ times as few distinct pixel values.
\item<2-> Note also the vertical axis scaling in $h_B(l)$.
\end{itemize}
\end{column}
\end{columns}
\end{slide}

\subsection{Stretched/compressed pixel value ranges}
%%%%%%%%%%%%%%%%%%%%%%%%%%%%%%%%%%%
%
%
%%%%%%%%%%%%%%%%%%%%%%%%%%%%%%%%%%%
\begin{slide}{Stretched/compressed pixel value ranges}
\begin{titlelist}{}{}

\item<2-> $B(i,j)=g(A(i,j))$\\
Suppose $g(l)$ represents an {\em overall} point function which
includes contrast stretching/compression, emphasis/de-emphasis, rounding,
normalizing etc.

\item<3-> Given an image matrix ${\bf A}$, $B(i,j)=g(A(i,j))$ is also
an image matrix.

\item<4-> $g(l)$ may not be ``continuous'' or connected and it also
may not be composed of connected line segments.

\item<5-> How do we determine which pixel value ranges $g(l)$ 
stretches/compresses?
\begin{itemize}
\item<6-> 
We can usually assume that in small ranges $g(l)$ may be
approximated by piecewise linear, connected line segments.
Calculating the implied $\alpha_i$ and testing $|\alpha_i| \stackrel{>}{<} 1$
should help us determine stretched/compressed ranges.
\end{itemize}

\end{titlelist}

\end{slide}


\subsection{Segmentation}
%%%%%%%%%%%%%%%%%%%%%%%%%%%%%%%%%%%
%
%
%%%%%%%%%%%%%%%%%%%%%%%%%%%%%%%%%%%
\begin{slide}{Segmentation}
\begin{titlelist}{}{}
\small
\item<2-> If one views an image as depicting a scene
composed of different objects, regions, etc.
then \red{segmentation} is the 
decomposition of an image into these objects and regions 
by associating or ``labelling'' each pixel with the object that it corresponds to.

\item<3-> Most humans can easily segment an image. 

\item<4-> Computer automated segmentation is a difficult problem,
requiring sophisticated algorithms that work in tandem.

\item<5-> ``High level'' segmentation, such as segmenting humans,
cars etc., from an image is a very difficult problem.
It is still considered unsolved and is actively researched.

\item<6-> Based on point processing, histogram based image segmentation is a very simple algorithm
that is sometimes utilized as an initial guess at the ``true''
segmentation of an image. 

\end{titlelist}

\end{slide}



%%%%%%%%%%%%%%%%%%%%%%%%%%%%%%%%%%%%%%%%%%%%%%%%%%%%%%%%%%%%%%%%%%%%%%%%
%
%
%%%%%%%%%%%%%%%%%%%%%%%%%%%%%%%%%%%%%%%%%%%%%%%%%%%%%%%%%%%%%%%%%%%%%%%%
\section{Histogram Based Image Segmentation}

\subsection{Methods}
%%%%%%%%%%%%%%%%%%%%%%%%%%%%%%%%%%%
%
%
%%%%%%%%%%%%%%%%%%%%%%%%%%%%%%%%%%%
\begin{slide}{Histogram Based Image Segmentation}
\begin{titlelist}{}{}

\item<2-> For a given image, decompose the range of pixel values $(0,\ldots, 255)$ into
``discrete'' intervals $R_t=[a_t,b_t],\ t=1,\ldots,T$, where $T$ is the total
number of segments. 

\item<3-> 
Each $R_t$ is typically obtained as a range of pixel
values that correspond to a \blue{hill} of $h_A(l)$.

\item<4-> 
``Label'' the pixels with pixel values within each $R_t$
via a point function. 

\item<5-> 
\red{Main Assumption:} Each object is assumed
to be composed of pixels with {\em similar} pixel values.

\end{titlelist}
\end{slide}


\subsection{Example}
%%%%%%%%%%%%%%%%%%%%%%%%%%%%%%%%%%%
%
%
%%%%%%%%%%%%%%%%%%%%%%%%%%%%%%%%%%%
\begin{slide}{Example}
\vspace{-6ex}
\slidefig[0.85]{claire.jpg}
\vspace{-6ex}
\begin{titlelist}{}{}
\small
\item<2-> 
$R_1=[0,14],R_2=[15, 15],R_3=[16, 99],
R_4=[100,149],R_5=[150,220],R_6=[221,255]$.

\item<3-> 
Labeling in matlab: $>> B1=255*((A>=0)\&(A<=14));$, etc.

\item<4-> 
This is what we did in the histogram slicing lecture last week

\end{titlelist}
\end{slide}


%%%%%%%%%%%%%%%%%%%%%%%%%%%%%%%%%%%
%
%
%%%%%%%%%%%%%%%%%%%%%%%%%%%%%%%%%%%
\begin{slide}{Example}
\vspace{-3ex}
\slidefig[0.775]{hist1.jpg}
\end{slide}

%%%%%%%%%%%%%%%%%%%%%%%%%%%%%%%%%%%
%
%
%%%%%%%%%%%%%%%%%%%%%%%%%%%%%%%%%%%
\begin{slide}{Example}
\vspace{-3ex}
\slidefig[0.775]{hist2.jpg}
\end{slide}

%%%%%%%%%%%%%%%%%%%%%%%%%%%%%%%%%%%
%
%
%%%%%%%%%%%%%%%%%%%%%%%%%%%%%%%%%%%
\begin{slide}{Example}
\vspace{-3ex}
\slidefig[0.8]{hist3.jpg}
\begin{titlelist}{}{}
\small
\item<2-> Compute the sample mean of each segment \\
($>> m1=sum(sum(B1.*A))/sum(sum(B1))$, etc.).

\item<3-> $C=m1\times B1+m2\times B2+m3\times B3+m4\times B4+
m5\times B5+m6\times B6$.\\
$B(i,j)=g_s^C(C(i,j))$.


\end{titlelist}
\end{slide}


\subsection{Limitations}
%%%%%%%%%%%%%%%%%%%%%%%%%%%%%%%%%%%
%
%
%%%%%%%%%%%%%%%%%%%%%%%%%%%%%%%%%%%
\begin{slide}{Limitations}
\begin{titlelist}{}{}

\item<2-> Histogram based segmentation
operates on each image pixel independently.
As mentioned earlier, the main
assumption is that objects must be composed
of pixels with similar pixel values.

\item<3-> This independent processing ignores
a second important property: Pixels
within an object should be {\em spatially} connected.  
For example, B3, B4, B5 group spatially disconnected
objects/regions into the same segment.

\item<4-> In practice, one would use histogram based segmentation
in tandem with other algorithms
that make sure that computed objects/regions
are spatially connected.

\end{titlelist}

\end{slide}




%%%%%%%%%%%%%%%%%%%%%%%%%%%%%%%%%%%%%%%%%%%%%%%%%%%%%%%%%%%%%%%%%%%%%%%%
%
%
%%%%%%%%%%%%%%%%%%%%%%%%%%%%%%%%%%%%%%%%%%%%%%%%%%%%%%%%%%%%%%%%%%%%%%%%
\section{Histogram Equalization}
%%%%%%%%%%%%%%%%%%%%%%%%%%%%%%%%%%%
%
%
%%%%%%%%%%%%%%%%%%%%%%%%%%%%%%%%%%%
\begin{slide}{Histogram Equalization}
\begin{titlelist}{}{}
\small
\item<2-> For a given image ${\bf A}$, we will now
design a special point function \darkred{$g_A^e(l)$} which
is called \darkred{the histogram equalizing point function
for ${\bf A}$}.

\item<3-> If $B(i,j)=g_A^e(A(i,j))$, then our aim is to make
$h_B(l)$ \blue{as uniform/flat as possible {\em irrespective}
of $h_A(l)$!}

\item<4-> Histogram equalization will help us:
\begin{itemize}\scriptsize
\item<5-> Stretch/Compress an image such that:
	\begin{itemize}\scriptsize
	\item Pixel values that occur frequently in ${\bf A}$ occupy
	a bigger dynamic range in ${\bf B}$, i.e., get stretched and become more
	visible.
	\item Pixel values that occur infrequently in ${\bf A}$ occupy
	a smaller dynamic range in ${\bf B}$, i.e., get compressed and become less
	visible.
	\end{itemize}
\item<6-> Compare images by ``mapping'' their histograms into a standard
histogram and sometimes ``undo'' the effects of some unknown processing.
\end{itemize}

\item<7-> The techniques we are going to use to get $g_A^e(l)$
are also applicable in histogram modification/specification.

\end{titlelist}
\end{slide}


\subsection{Random Variable}
%%%%%%%%%%%%%%%%%%%%%%%%%%%%%%%%%%%
%
%
%%%%%%%%%%%%%%%%%%%%%%%%%%%%%%%%%%%
\begin{slide}{Random Variables (RV)}
\begin{titlelist}{}{}
\small
\item<2-> A random variable, $\chi$, is a real-valued function defined on the events of the sample space, $S$.  

\item<3-> In other words, for each event in $S$, there is a real number that is the corresponding value of the random variable.  

\item<4-> Viewed yet another way, a \blue{random variable maps each event in $S$ onto the real line}. 

\item<5-> Example:
\begin{itemize}

\item<6-> 
consider the experiment of throwing a single die and observing the value of the up-face.  

\item<7-> 
We can define a random variable as the numerical outcome of the experiment (i.e., 1 through 6), but there are many other possibilities.  

\item<8-> 
For example, a binary random variable could be defined simply by letting $\chi = 0$ represent the event that the outcome of a throw is an even number and $\chi=1$ otherwise.
%represent the event that the outcome of throw is an even number and $\chi = 1$ otherwise.

\end{itemize}
%\item<9-> 
%\item<10-> 

\end{titlelist}

\end{slide}

%%%%%%%%%%%%%%%%%%%%%%%%%%%%%%%%%%%
%
%
%%%%%%%%%%%%%%%%%%%%%%%%%%%%%%%%%%%
\begin{slide}{Random Variables (RV)}
\begin{titlelist}{}{}
\small
\item<2-> In the discrete case, the probabilities of events are numbers between 0 and 1.  

\item<3-> Later we shall see that the gray levels in an image may be viewed as random variables in the interval $[0,1]$

\item<4-> Probability Density Function (PDF) is one of the fundamental descriptors of a random variable


\item<5-> 
However when dealing with continuous quantities (which are not denumerable) we can no longer talk about the "probability of an event" because that probability is zero. 

\item<6-> 
Thus in continuous case, instead of talking about the probability of a specific value, we talk about the probability that the value of the random variable lies in a specified range.  
\begin{itemize}
\scriptsize
\item<7-> 
In particular, we are interested in the probability that the random variable is less than or equal to (or, similarly, greater than or equal to) a specified constant $a$.  

\item<8-> 
Usually this is written as $ F_{\chi}(a) = P \left( \chi \leq a\right) $
\end{itemize}
%represent the event that the outcome of throw is an even number and $\chi = 1$ otherwise.

%\item<9-> 
%\item<10-> 

\end{titlelist}

\end{slide}



%%%%%%%%%%%%%%%%%%%%%%%%%%%%%%%%%%%
%
%
%%%%%%%%%%%%%%%%%%%%%%%%%%%%%%%%%%%
\begin{slide}{Transformation of Random Variable}
\begin{titlelist}{}{}

\item<2-> 
Suppose that a random variable $\chi$ is transformed by a monotonic transformation function $T(\chi)$ to produce a new random variable $Y$, 

\item<3-> 
Then the probability density function of $Y$ can be obtained from knowledge of $T(\chi)$ and the probability density function of $\chi$, as follows:
\vspace{-2ex}
\[ f_{Y}(y) = f_{\chi}(x) \left| \dfrac{dx}{dy} \right| \]

\item<4-> Applied to image:


\begin{center}
\onslide<5->
\mbox{The PDF of the transformed variable (output image)}\\
\onslide<6->
\mbox{is determined by}\\
\onslide<7->
\mbox{The PDF of the input image; and}\\
\onslide<8->
\mbox{the chosen transformation function}
\end{center}

\end{titlelist}

\end{slide}

\subsection{Continuous Amplitude RV}
%%%%%%%%%%%%%%%%%%%%%%%%%%%%%%%%%%%
%
%
%%%%%%%%%%%%%%%%%%%%%%%%%%%%%%%%%%%
\begin{slide}{Continuous Amplitude Random Variables}
\scriptsize
\begin{itemize}
\item Let $\chi$ be a continuous amplitude random variable $\chi \in (-\infty,+\infty)$.

\begin{tabular}{ll}\scriptsize
$f_{\chi}(x)$: & the \darkred{probability density function} of $\chi$,\\
$F_{\chi}(x)$: & the \darkred{probability distribution function}
of $\chi$.
\end{tabular}

\begin{eqnarray}
f_{\chi}(x)dx&=&\mbox{Probability}(x\leq \chi < x+dx)\\
F_{\chi}(x)&=&\mbox{Probability}(\chi \leq x)
\label{eq:Fdef}
\end{eqnarray}
\item Properties:
\begin{eqnarray}
F_{\chi}(x)&=&\int_{-\infty}^{x} f_{\chi}(t)dt  \Rightarrow  {dF_{\chi}(x)\over
dx} = f_{\chi}(x)
\label{eq:deriv}
\end{eqnarray}
\begin{eqnarray}
f_{\chi}(x)&\geq&0  \Rightarrow  F_{\chi}(x)\geq 0,\ F_{\chi}(x+dx)-F_{\chi}(x)\geq 0
\label{eq:incr}
\end{eqnarray}
$F_{\chi}(x)$ is a non-decreasing function.
\begin{eqnarray}
\int_{-\infty}^{+\infty} f_{\chi}(t)dt &=&1 \Rightarrow
f_{\chi}(x)|_{x=+/-\infty}=0 \\
F_{\chi}(x)|_{x=+\infty}&=&1 \\
F_{\chi}(x)|_{x=-\infty}&=&0
\end{eqnarray}

\end{itemize}
\end{slide}


%%%%%%%%%%%%%%%%%%%%%%%%%%%%%%%%%%%
%
%
%%%%%%%%%%%%%%%%%%%%%%%%%%%%%%%%%%%
\begin{slide}{Examples}
\begin{columns}
\begin{column}{0.7\textwidth}
\slidefig[0.6]{prob.jpg}
\end{column}
\begin{column}{0.3\textwidth}\scriptsize
Gaussian: \\
$f_{\chi}(x)={1\over \sqrt{2\pi \sigma^2}}e^{-(x-\mu)^2\over 2\sigma^2}$\\
\vspace{15ex}
\hypertarget{unif}{Uniform} $(a<b)$:
$$f_{\chi}(x)=\left\{\begin{array}{ll}
{1\over b-a} & a<x<b \\
0 & \mbox{otherwise}
\end{array}\right.$$
\end{column}
\end{columns}
\end{slide}


%%%%%%%%%%%%%%%%%%
%
%
%%%%%%%%%%%%%%%%%%
\begin{slide}{Calculating the Mean and Variance}
\begin{itemize}
\item Mean $(\darkred{\mu})$:
\begin{eqnarray}
\darkred{\mu}=\int_{-\infty}^{+\infty}x f_{\chi}(x)dx
\end{eqnarray}
\blue{Analogy:} Average price of apples \\
\begin{itemize}
\item ``I  bought $f_{\chi}(x)dx$ many apples at a price of $x$, ...''
\item ``Total price I paid: $P=\int_{-\infty}^{+\infty}x f_{\chi}(x)dx$.
\item ``Total number of apples I purchased: 
$N=\int_{-\infty}^{+\infty}f_{\chi}(x)dx=1$.
\item ``My average price for the overall purchase: $\mu=P/N$.
\end{itemize}
\item Variance $(\darkred{\sigma^2})$:
\begin{eqnarray}
\darkred{\sigma^2}=\int_{-\infty}^{+\infty}(x-\mu)^2 f_{\chi}(x)dx
\end{eqnarray}
\end{itemize}

\end{slide}


%%%%%%%%%%%%%%%%%%
%
%
%%%%%%%%%%%%%%%%%%
\begin{slide}{Main Derivation}
\begin{titlelist}{}{}

\item<2-> 
We will now obtain a new random variable $Y$ $(f_Y(y),F_Y(y))$ \red{as
a {\em function} of the random variable $\chi$}, i.e., $Y=g(\chi)$.

\item<3-> 
We wish to make $Y$ a uniform random variable
(a random variable having the uniform probability density
function) \blue{irrespective} of the density of $\chi$.

\item<4-> 
Our main assumption will be:

\begin{itemize}
\item<5-> 
Assume $F_{\chi}(x)$ is a continuous and strictly increasing function
(compared to the general case of non-decreasing as in Equation~\eqref{eq:incr})
\item<6-> Note that such an $F_{\chi}(x)$ is one-to-one which
will allow us to use its inverse $F_{\chi}^{-1}(x)$.
\end{itemize}

\end{titlelist}

\end{slide}



%%%%%%%%%%%%%%%%%%
%
%
%%%%%%%%%%%%%%%%%%
\begin{slide}{Main Derivation (cont'd)}

\begin{titlelist}{}{}
\scriptsize
\item<2-> 
Let $Y=F_{\chi}(\chi)$, i.e., $g(\chi)=F_{\chi}(\chi)$.
Note that $Y \in [0,1]$ and $f_Y(y)=0$ if $y\not\in [0,1]$.

\item<3-> 
Let us derive $F_Y(y)$ for $y\in [0,1]$:
\begin{eqnarray*}
F_{Y}(y)&=&\mbox{Probability}(Y \leq y)\\
&=&\mbox{Probability}(F_{\chi}(\chi) \leq y)\\
&=&\mbox{Probability}(\chi \leq F_{\chi}^{-1}(y))\\
&=&F_{\chi}(F_{\chi}^{-1}(y)) \\
&=&y 
\end{eqnarray*}
where the next to last step follows from Equation \eqref{eq:Fdef}.

\item<4-> 
Using Equation \eqref{eq:deriv}  and $f_Y(y)=0$ if $y\not\in [0,1]$,
we have
\begin{eqnarray}
f_Y(y)=\left\{\begin{array}{ll}
0 & y<0,\ y>1 \\
1 & y\in [0,1]
\end{array}\right.
\end{eqnarray}
i.e., $Y$ is a uniform random variable with \hyperlink{unif}{$a=0$ and $b=1$} .

\end{titlelist}


\end{slide}

\subsection{Discrete Amplitude RV}


%%%%%%%%%%%%%%%%%%
%
%
%%%%%%%%%%%%%%%%%%
\begin{slide}{Discrete Amplitude Random Variables}
\begin{itemize}
\scriptsize
\item Let $\Theta$ be a discrete amplitude random variable.\\
$\Theta=x_i$ for some $i$, $\ldots,-1,0,1,\ldots$.\\
$x_i$ are a sequence of possible values for $\Theta$.
\vspace{.2in}
 
\begin{tabular}{ll}
$p_{\Theta}(x_i)$: & the \darkred{probability mass function} of $\Theta$,\\
$F_{\Theta}(x_i)$: & the \darkred{probability distribution function}
of $\Theta$.
\end{tabular}
\begin{eqnarray}
p_{\Theta}(x_i)&=&\mbox{Probability}(\Theta=x_i)\\
F_{\Theta}(x_i)&=&\mbox{Probability}(\Theta \leq x_i)
\end{eqnarray}
\item Properties:
\begin{eqnarray}
F_{\Theta}(x_i)&=&\sum_{j=-\infty}^{j=i} p_{\Theta}(x_j) \\
p_{\Theta}(x_i)&=&F_{\Theta}(x_i)-F_{\Theta}(x_{i-1}) \geq 0\\
\sum_{j=-\infty}^{j=+\infty} p_{\Theta}(x_j)&=&1
\end{eqnarray}
\end{itemize}

\end{slide}



%%%%%%%%%%%%%%%%%%
%
%
%%%%%%%%%%%%%%%%%%
\begin{slide}{Example}
\slidefig[0.8]{upmf.jpg}

The probability mass and distribution functions for a uniform,
discrete amplitude random variable.
\end{slide}



%%%%%%%%%%%%%%%%%%
%
%
%%%%%%%%%%%%%%%%%%
\begin{slide}{Derivation for \hypertarget{darv}{Discrete Amplitude R.V.s}}
\slidefig[0.75]{diseq.jpg}

\begin{itemize}\small
\item Let $\Omega=F_{\Theta}(\Theta)$.
$\Omega=y_i=F_{\Theta}(x_i)$ for some $i$, $\ldots,-1,0,1,\ldots$.
\item Our earlier derivation for continuous amplitude random
variables does not ``work'' for discrete amplitude random variables.
\item In general $\Omega$ is not a uniform random variable. 
\end{itemize}
\end{slide}


%%%%%%%%%%%%%%%%%%
%
%
%%%%%%%%%%%%%%%%%%
\begin{slide}{Histogram as a Probability Mass Function}
\begin{itemize}
\small
\item For a given image ${\bf A}$,
consider the image pixels as the realizations of
a discrete amplitude random variable ``$A$''.
\begin{itemize}\scriptsize
\item For example suppose we toss a coin {(Heads=$255$ and Tails=$0$)}
$N\times M$ times and record the results as an $N$ by $M$ image matrix.
\end{itemize}
\item Define the sample probability mass function $p_{A}(l)$
as the probability of a randomly chosen pixel having the value $l$.
\begin{eqnarray}
p_{A}(l)={h_A(l) \over NM}
\end{eqnarray}
\item Note that the sample mean and variance we talked about
in previous lecture can be calculated as:
\begin{eqnarray*}
m_A&=&\sum_{l=0}^{255} l p_A(l) \\
\sigma^2_A&=&\sum_{l=0}^{255} (l-m_A)^2 p_A(l)
\end{eqnarray*} 
\end{itemize}
\end{slide}

\subsection{Histogram equalizing point function}
%%%%%%%%%%%%%%%%%%
%
%
%%%%%%%%%%%%%%%%%%
\begin{slide}{Histogram equalizing point function}
\begin{titlelist}{}{}
\small
\item<2-> Let $g_1(l)=\sum_{k=0}^{l} p_A(k)$.\\
$
g_1(l)-g_1(l-1)=p_A(l)= {h_A(l)\over NM}
$.\\


Note that $g_1(l)\in [0,1]$.

\item<3-> 
\darkred{$g_A^e(l)=\mbox{round}(255 g_1(l))$}
is the histogram equalizing point function for the image ${\bf A}$.

\item<4-> 
Image ${\bf A}\ \Rightarrow$ ``equalize image'' $\Rightarrow
B(i,j)=g_A^e(A(i,j))$.

%\mynote{4}{24}{
%Note that A is an image matrix and
%so is B, thanks to the way we have constructed the histogram equalizing
%point function
%}

\item<5-> 
\hyperlink{darv}{As we have seen}, in general $p_B(l)$
will not be a uniform probability mass function but hopefully
it will be close. 

\item<6-> 
In matlab $>> \mbox{help filter}$ to construct
$g_A^e(A(i,j))$ fast.

\item<7-> 
Assuming $gAe$ is an array that contains the computed
$g_A^e(l)$, we can use $>> B=gAe(A+1);$ to obtain the equalized image.

\end{titlelist}

\end{slide}



%%%%%%%%%%%%%%%%%%
%
%
%%%%%%%%%%%%%%%%%%
\begin{slide}{Stretching and Compression}
\vspace*{-3ex}
\slidefig[0.7]{range.jpg}
\begin{itemize}
\small
\item $g_A^e(l)$ stretches the range of pixel values that occur
frequently in ${\bf A}$.
\item $g_A^e(l)$ compresses the range of pixel values that occur
infrequently in ${\bf A}$.
\end{itemize}

\end{slide}



%%%%%%%%%%%%%%%%%%
%
%
%%%%%%%%%%%%%%%%%%
\begin{slide}{Examples}
\vspace*{-3ex}
\slidefig[0.75]{lenae1.jpg}
\vspace*{-4ex}
\slidefig[0.6]{lenae2.jpg}
\end{slide}


%%%%%%%%%%%%%%%%%%
%
%
%%%%%%%%%%%%%%%%%%
\begin{slide}{\hypertarget{compare}{Comparison}}
\begin{columns}
\begin{column}{0.7\textwidth}
\vspace*{-3ex}
\slidefig[0.7]{eq1.jpg}
\vspace*{-6ex}
\end{column}
\begin{column}{0.3\textwidth}
\scriptsize
 Instead of comparing ${\bf A}$ and ${\bf C}$, compare their
equalized versions.
\end{column}
\end{columns}
\end{slide}

%%%%%%%%%%%%%%%%%%
%
%
%%%%%%%%%%%%%%%%%%
\begin{slide}{Comparison}
\begin{columns}
\begin{column}{0.7\textwidth}
\vspace*{-3ex}
\slidefig[0.7]{eq2.jpg}
\vspace*{-6ex}
\end{column}
\begin{column}{0.3\textwidth}
\scriptsize
 Instead of comparing ${\bf A}$ and ${\bf C}$, compare their
equalized versions.
\end{column}
\end{columns}
\end{slide}


%%%%%%%%%%%%%%%%%%%%%%%%%%%%%%%%%%%
%
%
%%%%%%%%%%%%%%%%%%%%%%%%%%%%%%%%%%%
\begin{slide}{Another example}
\vspace{-4ex}
\slidefig[0.5]{histeq1.jpg}
\end{slide}


%%%%%%%%%%%%%%%%%%%%%%%%%%%%%%%%%%%
%
%
%%%%%%%%%%%%%%%%%%%%%%%%%%%%%%%%%%%
\begin{slide}{Another example}
\vspace{-4ex}
\slidefig[0.5]{histeq2.jpg}
\end{slide}





%%%%%%%%%%%%%%%%%%
%
%
%%%%%%%%%%%%%%%%%%
\begin{slide}{Practical considerations}
\begin{bulletlist} %{}{}
\small
\item<2-> \textbf{Histogram calculation}
\begin{columns}
\begin{column}{0.5\textwidth}\scriptsize
Manual computation:
\begin{code}[8]{}
\begin{tabbing}
>>  \=  h = zeros(256,1); \\
>>	\>	for  \= l = 0:255   \\
	\>  \>   h(l+1) = sum(sum(A==l)); \\
	\>  end \\
>> \> bar(0:255,h); \% display \\
\end{tabbing}  
\end{code}
\end{column}
\begin{column}{0.5\textwidth}\scriptsize
Matlab built in function:
\begin{code}[8]{}
[h,bin] = imhist(A) \% calculate \\
imhist(A) \% display \\
\end{code}
\vfill
\end{column}
\end{columns}

\item<3-> \textbf{Histogram stretching}
\begin{columns}
\begin{column}{0.5\textwidth}
\begin{code}[8]{}
I = imread('image.jpg'); \\
a = min(I(:)); \\
b = max(I(:)); \\
J = 255 * (I-a)/(b-a); \\
J = unit8(J)
\end{code}
\end{column}
\begin{column}{0.5\textwidth}
\[
J = \dfrac{I - I_{\min}}{I_{\max} - I_{\min}}
\]
\end{column}
\end{columns}


\end{bulletlist}

\end{slide}


%%%%%%%%%%%%%%%%%%
%
%
%%%%%%%%%%%%%%%%%%
\begin{slide}{Practical considerations}
\begin{bulletlist} %{}{}
\small
\item<2-> \textbf{Histogram adjustment}
\begin{columns}
\begin{column}{0.5\textwidth}
\begin{code}[7]{}
I = imread('lizard.jpg'); \\
A = 75;  \% lower threshold \\
B = 155; \% upper threshold \\

\% Initialize \\
\% Areas for intensity >= B \\
K1 = zeros(size(I)); \\
\% Areas for A <= intensity < B \\
K2 = K1; \\
\% Areas for intensity < A \\
K3 = K1; \\

\% Calculation \\
\% labeling K1 and set its value to 255 \\
K1 = 255 * (I>=B); \\
\% selecting area K2 and normalize \\
K2 = 255 * ((I>=A) \& (I<B)) .* (double(I)-A)/(B-A); \\
\% labeling K3 \\
K3 = 255 * (I < A); \\

K = K1 + K2 + 0 * K3; \\
K = uint8(K); \\

\end{code}
\end{column}
\begin{column}{0.5\textwidth}
Non-linear operation
\[ 
K = \begin{cases}
255 & \mbox{if}\ I \geq B \\
255 \frac{I-A}{B-A} & \mbox{if}\ A \leq I < B \\
0 & \mbox{if}\ I < A
\end{cases}
 \]
 
Matlab built-in function
\begin{code}[8]{}
K = imadjust(I,[75 150]/255,[]);
\end{code}
\end{column}
\end{columns}


%\item<5-> 
%
%\item<6-> 
%
%\item<7-> 
%
%\item<8-> 
%
%\item<9-> 
%
%\item<10-> 

\end{bulletlist}

\end{slide}


%%%%%%%%%%%%%%%%%%
%
%
%%%%%%%%%%%%%%%%%%
\begin{slide}{Practical}
\vspace{-3ex}
\begin{columns}
\begin{column}{0.45\textwidth}
\begin{center}
Original
\includegraphics[width=0.8\linewidth]{../mat/lizard}\\
\includegraphics[width=0.6\linewidth]{../mat/Hist_lizard}
\end{center}
\end{column}
\begin{column}{0.45\textwidth}
\begin{center}
Histogram Adjusted
\includegraphics[width=0.8\linewidth]{../mat/liz_histadjust}\\
\includegraphics[width=0.6\linewidth]{../mat/Hist_liz_histadjust}
\end{center}
\end{column}
\end{columns}
\end{slide}


%%%%%%%%%%%%%%%%%%
%
%
%%%%%%%%%%%%%%%%%%
\begin{slide}{LUT: Look Up Table}
\begin{bulletlist} %{}{}
\small
\item<2-> \textbf{Look Up Table}
\begin{columns}
\begin{column}{0.5\textwidth}
\begin{code}[7]{}
J = rgb2gray(imread('pepper.jpg')); \\
Jmin = min(J(:));\\
Jmax = max(J(:));\\

\% create Look Up Table \\
L = zeros(1,256);\\
L(1:Jmin) = 0;\\
L(Jmin+1:Jmax) = linspace(0,255,Jmax-Jmin);\\
L(Jmax+1:255)=255;\\

Y = uint8(L(J+1));\\
imwrite(Y,'pepperlut.jpg','jpeg');\\

\end{code}
\end{column}
\begin{column}{0.5\textwidth}
\begin{center}
\includegraphics[width=0.9\linewidth]{../mat/LUT}
\end{center}

\end{column}
\end{columns}


%\item<5-> 
%
%\item<6-> 
%
%\item<7-> 
%
%\item<8-> 
%
%\item<9-> 
%
%\item<10-> 

\end{bulletlist}

\end{slide}


%%%%%%%%%%%%%%%%%%
%
%
%%%%%%%%%%%%%%%%%%
\begin{slide}{Practical}
\vspace{-3ex}
\begin{columns}
\begin{column}{0.45\textwidth}
\begin{center}
Original\\
\includegraphics[width=0.6\linewidth]{../mat/pepper}\\
\includegraphics[width=0.6\linewidth]{../mat/LUT_pepper}
\end{center}
\end{column}
\begin{column}{0.45\textwidth}
\begin{center}
Histogram Adjusted\\
\includegraphics[width=0.6\linewidth]{../mat/pepperlut}\\
\includegraphics[width=0.6\linewidth]{../mat/LUT_pepperlut}
\end{center}
\end{column}
\end{columns}
\end{slide}



%%%%%%%%%%%%%%%%%%%%%%%%%%%%%%%%%%%
%
%
%%%%%%%%%%%%%%%%%%%%%%%%%%%%%%%%%%%
\begin{slide}{Histogram Matching -- Specification}
\begin{titlelist}{}{}

\item<2-> 
Given {\em images} ${\bf A}$ and ${\bf B}$,
using point processing
we would like to generate an image 
${\bf C}$ \blue{from ${\bf A}$}
such that $h_C(l)\sim h_B(l),\ (l=0,\ldots, 255)$.

\item<3-> 
More generally, given an image ${\bf A}$
and a histogram $h_B(l)$ {(or sample probability mass function
$p_B(l)$)}, we would like to generate an image 
${\bf C}$ such that $h_C(l)\sim h_B(l),\ (l=0,\ldots, 255)$.

\item<4-> 
Histogram matching/specification enables us
to ``match'' the grayscale distribution in one
image to  the grayscale distribution in another image.

\end{titlelist}
\end{slide}



%%%%%%%%%%%%%%%%%%%%%%%%%%%%%%%%%%%
%
%
%%%%%%%%%%%%%%%%%%%%%%%%%%%%%%%%%%%
\begin{slide}{Derivation for Continuous Amplitude R.V.s}
\begin{titlelist}{}{}
\scriptsize
\item<2-> 
We have already seen that for a continuous amplitude random variable
$\chi$ with strictly increasing and continuous $F_{\chi}(x)$,
the random variable $Y=F_{\chi}(\chi)$ has the uniform
probability density/distribution function.

\item<3-> 
\green{{\em Equivalently}}, for a continuous amplitude random variable
$Y\in [0,1]$ which has the uniform probability density function,
$\chi=F_{\chi}^{-1}(Y)$ has the probability density [distribution] function
$f_{\chi}(x)$ $[F_{\chi}(x)]$.
\vspace{-3ex}
\begin{eqnarray*}
\chi \ [F_{\chi}(x)] &\Rightarrow &Y=F_{\chi}(\chi) \ [\mbox{ uniform}]\\
Y \ [\mbox{uniform}] &\Rightarrow &\chi=F_{\chi}^{-1}(Y) \ [F_{\chi}(x)]
\end{eqnarray*}

\item<4-> 
\red{Now,} assume we have a continuous amplitude random variable
$Z$ with strictly increasing and continuous $F_{Z}(z).$
Then:
\vspace{-2ex}
\begin{eqnarray}
Y \ (\mbox{uniform}) &\Rightarrow &W=F_{Z}^{-1}(Y) \ [F_{Z}(w)]
\end{eqnarray}
but we can generate the required uniform random variable $Y$
from $\chi$ via $Y=F_{\chi}(\chi)$ which means $W$ can be generated 
\blue{from $\chi$} via:
\vspace{-2ex}
\begin{eqnarray}
W=F_{Z}^{-1}(Y)=F_{Z}^{-1}(F_{\chi}(\chi))
\end{eqnarray}

\end{titlelist}
\end{slide}




%%%%%%%%%%%%%%%%%%
%
%
%%%%%%%%%%%%%%%%%%
\begin{frame}[allowframebreaks]{Assignment}
\begin{enumerate}
\item Implement normalisation to your image. 

\item Implement histogram based segmentation on your image.
Identify the peaks of your histogram with the ``objects'' that
they correspond to.
Show your image, its histogram, the ranges, etc.
Show the identified objects.
Finally construct the histogram segmented image.

%\item Derive the mean and variance for continuous amplitude Gaussian and uniform densities.

\item Equalize your image.
Show before and after images and histograms.
Is the histogram of the equalized image uniform?
Which regions got stretched/compressed?
(Be as accurate as possible)

%\item Implement the \hyperlink{compare}{Comparison} experiment example on your image. 
\end{enumerate}
\end{frame}
