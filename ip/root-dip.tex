%\PassOptionsToPackage{pdflatex}{hyperref}
\PassOptionsToPackage{
%%				pdflatex,     
%%				 pdfpagelayout = useoutlines,
%%                 bookmarks,
%%                 bookmarksopen = true,
%%                 bookmarksnumbered = true,
%%                 breaklinks = true,
%%                 linktocpage,
%%                 pagebackref,
%                 colorlinks = true,
%                 linkcolor = red,
%%                 urlcolor  = blue,
%%                 citecolor = red,
%%                 anchorcolor = green,
%%                 hyperindex = true,
%%                 hyperfigures
}{hyperref}	% hypertext package, must come last of loaded packages;

\newif\ifMakeHandout % MakeHandout by default is set to false

%!TEX HANDOUT OPTION
\MakeHandouttrue	% Comment this line to make slide with progression; uncomment for making handout

\ifMakeHandout
	\documentclass[aspectratio=169,handout]{beamer}		% handout
\else	
	\documentclass[aspectratio=169]{beamer}				% slides with progressions
\fi
\usepackage{etex}	% must be loaded to make xy working

\usepackage{alltt}
\usepackage{array}
\usepackage{color}
\usepackage[caption=false]{subfig} % caption = false to let fig number appears
%\usepackage{caption}
%\usepackage{subcaption}
\usepackage{ifthen,calc}
%\usepackage{multimedia}
%\usepackage{times,mathptmx}

\usepackage{url}
\usepackage{xmpmulti}
\usepackage[xINCLUDENOTES,BIBNAT]{optional} %used with handout mode
\opt{BIBNAT}{
\usepackage[round]{natbib} % numbers
}
\usepackage[all]{xy}
\usepackage{amsmath}
\usepackage{listings}

\usepackage{pdfpages}

%\usetheme{Copenhagen} %% USED in Manet as well
%\usetheme{AnnArbor}
%\usetheme{Berkeley}

%\usetheme{default}
%\usetheme{AnnArbor}
%\usetheme{Antibes}
%\usetheme{Bergen}
%\usetheme{Berkeley}
%\usetheme{Berlin}
%\usetheme{Boadilla}
%\usetheme{CambridgeUS}
%\usetheme{Copenhagen}
%\usetheme{Darmstadt}
%\usetheme{Dresden}
%\usetheme{Frankfurt}
%\usetheme{Goettingen}
%\usetheme{Hannover}
%\usetheme{Ilmenau}
%\usetheme{JuanLesPins}
%\usetheme{Luebeck}
%\usetheme{Madrid}
%\usetheme{Malmoe}
%\usetheme{Marburg}
%\usetheme{Montpellier}
%\usetheme{PaloAlto}
%\usetheme{Pittsburgh}
%\usetheme{Rochester}
%\usetheme{Singapore}
%\usetheme{Szeged}
%\usetheme{Warsaw}

% As well as themes, the Beamer class has a number of color themes
% for any slide theme. Uncomment each of these in turn to see how it
% changes the colors of your current slide theme.

%\usecolortheme{albatross}
%\usecolortheme{beaver}
%\usecolortheme{beetle}
%\usecolortheme{crane}
%\usecolortheme{dolphin}
%\usecolortheme{dove}
%\usecolortheme{fly}
%\usecolortheme{lily}
%\usecolortheme{orchid}
%\usecolortheme{rose}
%\usecolortheme{seagull}
%\usecolortheme{seahorse}
%\usecolortheme{whale}
%\usecolortheme{wolverine}

\definecolor{LightGray}{gray}{0.95}
\definecolor{BlueTone}{rgb}{0.25,0.25,0.75}
\definecolor{DarkSeaGreen}{rgb}{0.50,0.74,0.50}

\definecolor{darkblue}{rgb}{0.25,0.25,0.75}

\setbeamerfont{block title}{size={},series=\bfseries}

\setbeamercolor{background canvas}{bg=white}
%\setbeamercolor{background canvas}{bg=LightGray}
%\setbeamercolor{alerted text}{fg=darkblue!80!yellow}
\setbeamercolor{alerted text}{fg=red,bg=yellow!90!orange}
\setbeamercolor{block title}{fg=yellow!90!orange,bg=darkblue}
\setbeamercolor{block title example}{fg=yellow!90!orange,bg=darkblue}

%\usetheme{Rochester}
%\definecolor{BlueTone}{rgb}{0.25,0.25,0.75}
%\setbeamercolor{frametitle}{bg=BlueTone}


\usetheme{Simple}


%%%%%%%%%%%%%%%%%%%%%%%%%%%%%%%%%%%%%%%%%%%%%%%%%%%%%%%%%%%%%%%%%%%%%%%%%%%%%%%%%%%%%%%%%%%%%%%%%%
%
%!TEX	CHAPTER SELECTION
%
%%%%%%%%%%%%%%%%%%%%%%%%%%%%%%%%%%%%%%%%%%%%%%%%%%%%%%%%%%%%%%%%%%%%%%%%%%%%%%%%%%%%%%%%%%%%%%%%%%

%\input root-par.tex
\newcommand{\chap}{00}        % Specify the chapter for which produce slides
%\setboolean{addnotes}{false}     % Uncomment if a booklet for adding notes is required - still Error
%\setboolean{includeall}{false} % Uncomment if a single booklet with all slides is needed

%\setbeameroption{show notes}

\newcommand{\Notes}{%
	\setlength{\arrayrulewidth}{0.1pt}
	\begin{tabular}{p{\textwidth}}\hline
	\\ \hline \\ \hline	\\ \hline \\ \hline \\ \hline
	\\ \hline \\ \hline	\\ \hline \\ \hline \\ \hline
	\\ \hline \\ \hline	\\ \hline \\ \hline \\ \hline
	\end{tabular}}

\newcommand{\emptybox}[2]{%
  \raisebox{0pt}[#1]{\makebox[#2]{\hspace*{#2}}}}

\newcommand{\addemptyslide}{\ifthenelse{\boolean{includeall}}{\begin{slide}{}\end{slide}}{}}

\newenvironment{slide}[1]{%
%\setbeamercolor{frametitle}{bg=yellow!85!orange,fg=darkblue}
  \begin{frame}{#1}% %% try with framebreaks: not working!!!
    \setbeamercovered{invisible} }{% CHECKING: transparent, invisible, dynamic
  \end{frame}
  \mode<handout>{%
  \opt{INCLUDENOTES}{  
  	\setbeamercolor{frametitle}{bg=white}
  	\addtocounter{framenumber}{-1}
  	\begin{frame}{}
    	\Notes
  	\end{frame}
  }% end of opt  
  }% end of ho
  \setbeamercolor{frametitle}{bg=yellow!85!orange}
  }

\newcommand{\emptyslide}{%
%\setbeamercolor{frametitle}{bg=yellow!85!orange}
  \mode<handout>{%
  \opt{INCLUDENOTES}{
  \setbeamercolor{frametitle}{bg=white}
  \addtocounter{framenumber}{-1}
  \begin{frame}{}
    \Notes
  \end{frame}
  } %end of opt
  } %end of ho
  \setbeamercolor{frametitle}{bg=yellow!85!orange}
}

\newcommand{\M}[1]{{${#1}$}}

\newcommand{\slidefig}[2][0.83]{%
	$$\includegraphics[scale=#1]{ch.\chap/#2}$$}

\newenvironment{bulletlist}{%
	\begin{itemize}\setlength{\itemsep}{3pt}}{%
	\end{itemize}}

\newenvironment{titlelist}[2]{%
  \begin{block}{#1}
    {#2}
    \begin{itemize}\setlength{\itemsep}{3pt}}{%
  \end{itemize}\end{block}}

\newenvironment{textlist}[1][xxx]{%
	\begin{description}[#1]\setlength{\itemsep}{0pt}}{%
	\end{description}}

\newcommand{\slideset}[1]{ch.#1/ip-slides.#1}
\newcommand{\includeslides}[1]{%
  \ifthenelse{\equal{#1}{\chap}}{%
    \include{\slideset{#1}}}{}}

\newcommand{\assignmentset}[1]{ch.#1/ip-assignment.#1}
\newcommand{\includeassignment}[1]{%
  \ifthenelse{\equal{#1}{\chap}}{%
    \include{\assignmentset{#1}}}{}}

\newcommand{\Red}[1]{{\color{red}#1}}
\newcommand{\red}[1]{{\color{red}#1}}
\newcommand{\Blue}[1]{{\color{blue}#1}}
\newcommand{\blue}[1]{{\color{blue}#1}}
\newcommand{\green}[1]{{\color{green}#1}}

\definecolor{darkred}{rgb}{0.6,0,0}
\newcommand{\darkred}[1]{{\color{darkred}#1}}

\newcommand{\PBS}[1]{\let\temp=\\#1\let\\=\temp}
\newcommand{\RRCOL}{\PBS\raggedright\hspace{0pt}}

\newcommand{\question}[2]{%
  \begin{exampleblock}{Question}{#1}\end{exampleblock}}

\newcommand{\zeroskip}{\setlength{\parskip}{0pt}}
\newcommand{\zerosep}{\setlength{\itemsep}{0pt}}

\newenvironment{code}[2][12]{%
  \fontsize{#1}{#1pt}\selectfont
  \begin{alltt}}{%
  \end{alltt}}

%---------------------------------------------------------

\newcommand{\TBD}[1]{\textbf{TBD: #1}}

\newcommand{\newterm}[1]{#1}
\newcommand{\id}[1]{\mbox{\emph{#1}}}
\renewcommand{\emph}[1]{{\itshape #1}}
\newcommand{\ts}[1]{\mbox{\itshape\scriptsize #1}}
\newcommand{\func}[1]{\mbox{\emph{#1}}}

%----------------------------------------------------------

% Hyperref

%\def\begincolor#1{\special{pdf:bc #1}}%
%\def\endcolor{\special{pdf:ec}}%
%\def\colored#1#2{%
%  \begincolor{#1}#2\endcolor}
%\def\linkcolor{[0.5 0.0 0.7]}%
%\def\titlecolor{[.04 .3 .6]}
%\def\exampcolor{[.8 0.3 0.2]}
%
%\def\setlink#1{\colored{\linkcolor}{#1}}%
%
%\def\dest#1{\special{pdf:dest (#1) [ @thispage /FitH
%@ypos ]}}
%
%\def\link#1#2{\setbox0\hbox{\setlink{#1}}%
%  \leavevmode\special{pdf: ann width \the\wd0\space\space height \the\ht0\space\space depth \the\dp0
%     << /Type /Annot /Subtype /Link /Border [ 0 0 0 ] /A << /S /GoTo
%     /D (#2) >> >>}\box0\relax}

\newcommand{\mynote}[3]{
\special{pdf: annotate width #1in height #2pt
  << /Type /Annot /Subtype /Text /Name /Help
     /Contents (#3) >>}
  %  << /Type /Annot /Subtype /FileAttachment /FS
  %  << /Type /Filespec /FS (command.com) /Dos (transistor.html) /EF
  %  << /Type /Filespec /Dos (transistor.html)>> >>
  %  >>}
  }

\renewcommand{\thepage}{{\hspace{-.0in} \large \copyright\ 
IPG}
\hfill \arabic{page}}

\newcommand{\slidefiganim}[3][1,height=0.2\textheight]{%
	{	\centering
		\includegraphics<#1>[#2]{ch.\chap/#3}
	}
}

\newcommand{\insertfig}[2][height=0.5\textheight]{%
	\includegraphics[#1]{ch.\chap/#2}}


%%%%%%%%%%%%%%%%%%%%%%%%%%%%%%%%%%%%%%%%%%%%%%%%%%%%%%%%%%%
%
% Notes: 
%	- topics based on TIT's lectures
%	- some materials based on poly's lectures
%
%%%%%%%%%%%%%%%%%%%%%%%%%%%%%%%%%%%%%%%%%%%%%%%%%%%%%%%%%%%
\newcommand{\topicslabel}{}
\newcommand{\czero}{}%% IPG
\newcommand{\ci}{Introduction}
\newcommand{\cii}{Image Formation}
\newcommand{\ciii}{Basic Operations}
\newcommand{\civ}{Histogram Equalisation}
\newcommand{\cv}{Convolution and Fourier Transform}
\newcommand{\cvi}{Smoothing \& Sharpening}
\newcommand{\cvii}{Edge Detection}
\newcommand{\cviii}{Image Segmentation}
\newcommand{\cix}{Image Compression}
\newcommand{\cx}{Steganography}
\newcommand{\cxi}{Fidelity Criteria}
\newcommand{\cxii}{}
\newcommand{\cxiii}{}
\newcommand{\cxiv}{}



\newtheorem{algorithm}{Algorithm}

%----------------------------------------------------------

\title{Image Processing (TIF356)}% 
\author{Irwan Prasetya Gunawan, Ph.D}
\date{}
\institute{Informatics, Bakrie University\\
           irwan.gunawan@bakrie.ac.id}


\setbeamertemplate{navigation symbols}{}
\setbeamertemplate{footline}[frame number]
\setbeamertemplate{caption}[numbered]


\begin{document}

\begin{frame}[plain]
  
  \titlepage


  \begin{center}\Large\color{blue}\vspace*{-36pt}\sloppy
    \ifthenelse{\equal{00}{\chap}}{\topicslabel 00: \czero\\}{}
    \ifthenelse{\equal{01}{\chap}}{\topicslabel 01: \ci\\}{}
    \ifthenelse{\equal{02}{\chap}}{\topicslabel 02: \cii\\}{}
    \ifthenelse{\equal{03}{\chap}}{\topicslabel 03: \ciii\\}{}
    \ifthenelse{\equal{04}{\chap}}{\topicslabel 04: \civ\\}{}
    \ifthenelse{\equal{05}{\chap}}{\topicslabel 05: \cv\\}{}
    \ifthenelse{\equal{06}{\chap}}{\topicslabel 06: \cvi\\}{}
    \ifthenelse{\equal{07}{\chap}}{\topicslabel 07: \cvii\\}{}
    \ifthenelse{\equal{08}{\chap}}{\topicslabel 08: \cviii\\}{}
    \ifthenelse{\equal{09}{\chap}}{\topicslabel 09: \cix\\}{}
    \ifthenelse{\equal{10}{\chap}}{\topicslabel 10: \cx\\}{} 
    \ifthenelse{\equal{11}{\chap}}{\topicslabel 11: \cxi\\}{} 
%    \ifthenelse{\equal{12}{\chap}}{\topicslabel 12: \cxii\\}{} 
%    \ifthenelse{\equal{13}{\chap}}{\topicslabel 13: \cxiii\\}{} 
%    \ifthenelse{\equal{14}{\chap}}{\topicslabel 14: \cxiv\\}{}
  \end{center}

  \begin{center}
  \scriptsize
    \red{Version: \today}
  \end{center}

%\vspace*{6ex}

\vfill

\begin{center}
\tiny
%Prepared for ... \\
\end{center}


  %{\vfill\hfill\includegraphics[width=8cm]{griffon}}

\end{frame}

%\setbeamercolor{frametitle}{bg=yellow!85!orange,fg=BlueTone}
%\mode<handout>{%
%\opt{INCLUDENOTES}{%
%  \setbeamercolor{frametitle}{bg=white}
%  \addtocounter{framenumber}{-1}
%  \begin{frame}{}
%    \Notes
%  \end{frame}
%  } %end of opt
%%  \setbeamercolor{frametitle}{bg=yellow!85!orange}
%\setbeamercolor{frametitle}{fg=yellow!85!orange,bg=darkblue}
%} %end of mode

%-----------------------------------------------------------------------

%\ifthenelse{\equal{00}{\chap}}{}{%
\begin{frame}{Contents}

\begin{center}\small
  \renewcommand{\arraystretch}{1.1}
  \begin{tabular}{|l|} \hline
%    \textbf{\topicslabel} \\ \hline
    % \ifthenelse{\equal{00}{\chap}}{\Red{00: \czero}} {00: \czero}  \\ \hline
    \ifthenelse{\equal{01}{\chap}}{\Red{01: \ci}}    {01: \ci}     \\ \hline
    \ifthenelse{\equal{02}{\chap}}{\Red{02: \cii}}   {02: \cii}    \\ \hline
    \ifthenelse{\equal{03}{\chap}}{\Red{03: \ciii}}  {03: \ciii}   \\ \hline
    \ifthenelse{\equal{04}{\chap}}{\Red{04: \civ}}   {04: \civ}    \\ \hline
    \ifthenelse{\equal{05}{\chap}}{\Red{05: \cv}}    {05: \cv}     \\ \hline
    \ifthenelse{\equal{06}{\chap}}{\Red{06: \cvi}}   {06: \cvi}    \\ \hline
    \ifthenelse{\equal{07}{\chap}}{\Red{07: \cvii}}  {07: \cvii}   \\ \hline
    \ifthenelse{\equal{08}{\chap}}{\Red{08: \cviii}} {08: \cviii}  \\ \hline
    \ifthenelse{\equal{09}{\chap}}{\Red{09: \cix }}  {09: \cix}    \\ \hline
    \ifthenelse{\equal{10}{\chap}}{\Red{10: \cx}}    {10: \cx}     \\ \hline
    \ifthenelse{\equal{11}{\chap}}{\Red{11: \cxi}}   {11: \cxi}    \\ \hline
%    \ifthenelse{\equal{12}{\chap}}{\Red{12: \cxii}}  {12: \cxii}   \\ \hline
%    \ifthenelse{\equal{13}{\chap}}{\Red{13: \cxiii}} {13: \cxiii}  \\ \hline
%    \ifthenelse{\equal{14}{\chap}}{\Red{14: \cxiv}}  {14: \cxiv}   \\ \hline
  \end{tabular}
\end{center}
  
\end{frame}


%\setbeamercolor{frametitle}{bg=yellow!85!orange,fg=BlueTone}
%\mode<handout>{%
%\opt{INCLUDENOTES}{%
%  \setbeamercolor{frametitle}{bg=white}
%  \addtocounter{framenumber}{-1}
%  \begin{frame}{}
%    \Notes
%  \end{frame}
%  } %end of opt
%%  \setbeamercolor{frametitle}{bg=yellow!85!orange}
%  \setbeamercolor{frametitle}{fg=yellow!85!orange,bg=darkblue}
%} %end of mode

%}

\includeslides{00}
\includeslides{01} 
\includeslides{02} 
\includeslides{03}
\includeslides{04}
\includeslides{05}
\includeslides{06}
\includeslides{07}
\includeslides{08}
\includeslides{09}
\includeslides{10}
\includeslides{11}
%\includeslides{12}
%\includeslides{13}
\includeslides{99}

\includeassignment{01}
\includeassignment{02}

%%%%%%%%%%%%%%%%%%
%
%
%%%%%%%%%%%%%%%%%%
\begin{slide}{References}
\nocite{*}
\opt{BIBNAT}{
\def\newblock{}
\bibliography{dip}
\bibliographystyle{plainnat}
} 
\opt{noBIBNAT}{
\bibliography{dip}
\bibliographystyle{plain}
} %alpha,plain,unsrt,abbrv
\end{slide}



\end{document}
